\section{实现简单传输层协议}

  完整的 TLS 1.3 协议栈支持需要大量研究的开发工作,比如 openssl 库,单人开发几乎是不可能的,在此使用浏览器/服务器模式实现简单的 TLS 1.3 协议,包括握手协议和记录层协议\cite{7958593}。利用 WebSocket 实现密钥协商功能、参数协商功能、建立共享密钥功能。WebSocket 是 HTML5提供的新功能,一种浏览器能与服务器之间进行全双工通信的网络技术,可以传输文本和二进制数据。

  \subsection{开发环境概述}

  \begin{enumerate}
    \item 服务器端:
    
      操作系统: OS X El Capitan

      服务器: Node.js v10.15.3

      数据库: Mongodb 和 Redis

      开发工具: Visual Studio Code

      浏览器: Chrome v73

    \item 客户端:

      开发工具: Wepack, Visual Studio Code

      浏览器: Chrome v73

  \end{enumerate}

  \subsection{开发语言概述}

    本次开发讲使用 HTML5、CSS3、JavaScript 等基本编程语言。HTML5 为我提供不少的浏览器接口,localStorage,Cookie 等存储方式,WebSocket 协议能使服务器主动推送消息到客户端,完成 TLS 1.3 握手功能。HTML5 为网页提供最基本骨架,页面的元素,让浏览器应用开发更加多样化,功能更强大的技术。各大浏览器支持情况良好。CSS3 作为 HTML5 的化妆师,提供页面美化和页面布局,样式化和排版前端网页,例如控制页面字体大小、颜色、间距等。CSS3选择器需要结合 HTML5 使用,在 HTML5 元素中使用对应的 CSS 选择器。JavaScript 是一种面向对象的动态语言,提供浏览器和用户进行页面交互的操作,是开发中的核心技术,在编程语言中排行前10。比较轻量级,插入到 HTML5 页面中,可以通过大多数浏览器解析执行。

  \subsection{功能介绍}

  实现握手协议,记录层协议。使用 WebSocket 交换客户端和服务器端的密钥参数,主要实现 AEAD 算法: AES-256-GCM, ChaCha20-Poly1305。HKDF\cite{RFC5869} 密钥导出算法。编程实现 FRC8446 中客户端和服务器端的状态机。完整的一轮往返握手和零轮往返,一轮往返握手为之后建立零轮往返建立基础,用于实现重放攻击。

  密钥交换信息。客户端发送的 ClientHello 中提供:

  \begin{enumerate}
    \item legacy\_version: 在 TLS 1.3 版本中必须设置成 0x0303。
    \item random: 安全随机数生成器产生的32字节随机数。
    \item legacy\_session\_id: 兼容模式下,这个值必须是非空,一个不可预测的值。
    \item cipher\_suites: 客户端支持的加密套件,TLS 1.3 中只支持5种加密套件。
    \item legacy\_compression\_methods: 必须设置为0的一个字节。
    \item extensions: 拓展
  \end{enumerate}

  服务器发送的 ServerHello 中提供:

  \begin{enumerate}
    \item legacy\_version: 在 TLS 1.3 版本中必须设置成 0x0303。
    \item random: 安全随机数生成器产生的32字节随机数。
    \item legacy\_session\_id\_echo: 兼容模式下,这个值必须是非空,一个不可预测的值。
    \item cipher\_suites: 从客户端支持的加密套件中选择加密套件。
    \item legacy\_compression\_methods: 必须设置为0的一个字节。
    \item extensions: 拓展。TLS 1.3 版本中必须包含 supported\_versions 扩展
  \end{enumerate}

  服务器参数和认证消息这里不展开详细说明。Certificate: 将证书链发送给对方,当约定的密钥交换方法是用证书进行认证的时候,服务器就必须发送 Certificate 消息,当且仅当客户端通过发送 CertificateRequest 消息请求认证客户端时,客户端必须发送 Certificate 消息,当客户端没有合适的证书时,必须发送不含证书的 Certificate 消息。Certificate Verify: 此消息用于证明发送方拥有其证书对应的私钥,必须在 Certificate 消息之后立即发送,并且紧接着在 Finished 消息之前。Finished: 提供握手和密钥的身份验证,Finished 消息的接受者必须验证内容是否正确,如果不正确,必须使用 decrypt\_error alert 消息终止连接。End of Early Data: 如果服务器在 EncryptedExtensions 中发送了 early\_data 扩展,则客户端必须在收到服务器的 Finished 消息后发送 EndOfEarlyData 消息。