\section{实现简单传输层协议}

  完整的 TLS 1.3 协议栈支持需要大量研究的开发工作,比如 openssl 库,单人开发几乎是不可能的,在此使用浏览器/服务器模式实现简单的 TLS 1.3 协议,包括握手协议和记录层协议\cite{7958593}。利用 WebSocket 实现密钥协商功能、参数协商功能、建立共享密钥功能。WebSocket 是 HTML5提供的新功能,一种浏览器能与服务器之间进行全双工通信的网络技术,可以传输文本和二进制数据。

  \subsection{开发环境概述}

  \begin{enumerate}
    \item 服务器端:
    
      操作系统: OS X El Capitan

      服务器: Node.js v10.15.3

      数据库: Mongodb 和 Redis

      开发工具: Visual Studio Code

      浏览器: Chrome v73

    \item 客户端:

      开发工具: Wepack, Visual Studio Code

      浏览器: Chrome v73

  \end{enumerate}

  \subsection{开发语言概述}

    本次开发讲使用 HTML5、CSS3、JavaScript 等基本编程语言。HTML5 为我提供不少的浏览器接口,localStorage,Cookie 等存储方式,WebSocket 协议能使服务器主动推送消息到客户端,完成 TLS 1.3 握手功能。HTML5 为网页提供最基本骨架,页面的元素,让浏览器应用开发更加多样化,功能更强大的技术。各大浏览器支持情况良好。CSS3 作为 HTML5 的化妆师,提供页面美化和页面布局,样式化和排版前端网页,例如控制页面字体大小、颜色、间距等。CSS3选择器需要结合 HTML5 使用,在 HTML5 元素中使用对应的 CSS 选择器。JavaScript 是一种面向对象的动态语言,提供浏览器和用户进行页面交互的操作,是开发中的核心技术,在编程语言中排行前10。比较轻量级,插入到 HTML5 页面中,可以通过大多数浏览器解析执行。

  \subsection{功能介绍}

  实现握手协议,记录层协议。使用 WebSocket 交换客户端和服务器端的密钥参数,主要实现 AEAD 算法: AES-256-GCM, ChaCha20-Poly1305。HKDF\cite{RFC5869} 密钥导出算法。编程实现 FRC8446 中客户端和服务器端的状态机。完整的一轮往返握手和零轮往返,一轮往返握手为之后建立零轮往返建立基础,用于实现重放攻击。

  密钥交换信息。客户端发送的 ClientHello 中提供:

  \begin{enumerate}
    \item legacy\_version: 在 TLS 1.3 版本中必须设置成 0x0303。
    \item random: 安全随机数生成器产生的32字节随机数。
    \item legacy\_session\_id: 兼容模式下,这个值必须是非空,一个不可预测的值。
    \item cipher\_suites: 客户端支持的加密套件,TLS 1.3 中只支持5种加密套件。
    \item legacy\_compression\_methods: 必须设置为0的一个字节。
    \item extensions: 拓展
  \end{enumerate}

  服务器发送的 ServerHello 中提供:

  \begin{enumerate}
    \item legacy\_version: 在 TLS 1.3 版本中必须设置成 0x0303。
    \item random: 安全随机数生成器产生的32字节随机数。
    \item legacy\_session\_id\_echo: 兼容模式下,这个值必须是非空,一个不可预测的值。
    \item cipher\_suites: 从客户端支持的加密套件中选择加密套件。
    \item legacy\_compression\_methods: 必须设置为0的一个字节。
    \item extensions: 拓展。TLS 1.3 版本中必须包含 supported\_versions 扩展
  \end{enumerate}

  服务器参数和认证消息这里不展开详细说明。Certificate: 将证书链发送给对方,当约定的密钥交换方法是用证书进行认证的时候,服务器就必须发送 Certificate 消息,当且仅当客户端通过发送 CertificateRequest 消息请求认证客户端时,客户端必须发送 Certificate 消息,当客户端没有合适的证书时,必须发送不含证书的 Certificate 消息。Certificate Verify: 此消息用于证明发送方拥有其证书对应的私钥,必须在 Certificate 消息之后立即发送,并且紧接着在 Finished 消息之前。Finished: 提供握手和密钥的身份验证,Finished 消息的接受者必须验证内容是否正确,如果不正确,必须使用 decrypt\_error alert 消息终止连接。End of Early Data: 如果服务器在 EncryptedExtensions 中发送了 early\_data 扩展,则客户端必须在收到服务器的 Finished 消息后发送 EndOfEarlyData 消息。

\subsection{密钥导出算法}

在 TLS 1.3 中,不再使用 PRF 算法,而是采用更标准的 HKDF 算法来进行密钥的推导。而且在 TLS 1.3 中对密钥进行了更细粒度的优化,每个阶段或者方向的加密都不是使用同一个密钥。TLS 1.3 在 ServerHello 消息之后的数据都是加密的,握手期间服务器给客户端发送的消息用 server\_handshake\_traffic\_secret 通过 HKDF 算法导出的密钥加密的,Client 发送给 Server 的握手消息是用 client\_handshake\_traffic\-\_secret 通过 HKDF 算法导出的密钥加密的。这两个密钥是通过 Handshake Secret 密钥来导出的,而 Handshake Secret 密钥又是由 PreMasterSecret 和 Early Secret 密钥导出,然后通过 Handshake Secret 密钥导出主密钥 Master Secret。

再由主密钥 Master Secret 导出这几个密钥: client\_application\_traffic\_secret:用来导出客户端发送给服务器应用数据的对称加密密钥。server\_application\_traffic\_s\-ecret:用来导出服务器发送给客户端应用数据的对称加密密钥。resumption\_master\_\\secret:用来生成 PSK。最终 server\_handshake\_traffic\_secret、client\_handshake\_traffic\_\\secret、 client\_application\_\-traffic\_secret、server\_application\_traffic\_secret 这 4 个密钥会分别生成 4 套 write\_key 和 write\_IV 用于对称加密。如果用到 early\_data,还需要 client\_e\-arly\_traffic\_secret,它也会生成 1 套 write\_key 和 write\_IV 用于加密和解密 0-RTT 数据。Key Derivation Function (KDF) 是密码学系统中必要的组件。它的目的是把一个 key 拓展成多个从密码学角度来上说是安全的 key。

TLS 1.3 使用的是 HMAC-based Extract-and-Expand Key Derivation Function 即 HKDF 函数,HKDF 根据 extract-then-expand 设计模式,即 KDF 有 2 大模块。第一个阶段是将输入的 key material 进行 "extracts",得到固定长度的 key,然后第二阶段将这个 key "expands" 成多个附加的伪随机的 key,输出的 key 的长度和个数,取决于指定的加密算法。由于 extract 流程不是必须的,所以 expand 流程可以独立的使用。HMAC 的两个参数,第一个是 key,第二个是 data。data 可以由好几个元素组成,一般用 | 来表示

经过密钥协商得出来的密钥材料的随机性可能不够,协商的过程能被攻击者获知,需要使用一种密钥导出函数来从初始密钥材料(PSK 或者 DH 密钥协商计算出来的 key)中获得安全性更强的密钥。HKDF 正是 TLS 1.3 中所使用的这样一个算法,使用协商出来的密钥材料和握手阶段报文的哈希值作为输入,可以输出安全性更强的新密钥。在 TLS 1.2 中使用的密钥导出函数 PRF 实际上只实现了 HKDF 的 expand 部分,并没有经过 extract,而直接假设密钥材料的随机性已经符合要求。因为 TLS 1.3 对密钥材料进行 extract\_then\_expand,所以这也是为什么 TLS 1.3 比 TLS 1.2 在安全性上更上一层楼的原因。TLS 1.3 中的所有密钥都是由 HKDF-Extract(salt, IKM) 和 Derive-Secret(Secret, Label, Messages) 联合导出的。其中 Salt 是当前的 secret 状态,输入密钥材料(IKM)是要添加的新 secret 。在 TLS 1.3 中,两个输入的 IKM 是: PSK 或者 (EC)DHE 共享 的 secret。一旦计算出了从给定 secret 派生出的所有值,就应该删除该 secret。

% TLS 1.3 中的 Finished 并不算是整个握手中的第一条加密消息,作用和 TLS 1.2 是相同的,它对提供握手和计算密钥的身份验证起了至关重要的作用。 Authentication 消息的计算统一采用以下的输入方式: 1. 要使用证书和签名密钥 2. 握手上下文由哈希副本中的一段消息集组成 3. Base key 用于计算 MAC 的密钥。Finished 子消息根据 Transcript-Hash(Handshake Context, Certificate, CertificateVerify) 的值得出的 MAC 。使用从 Base key 派生出来的 MAC key 计算的 MAC 值。用于计算 Finished 消息的密钥是使用 HKDF,Base Key 是 server\_handshake\_traffic\_ secret 和 client\_handshake\_traffic\_secret。

% 如果使用同一个密钥加密大量的数据,攻击者有几率可以通过记录所有密文并找出特征,逆推出对称加密密钥。因此需要引进一个密钥同步更新的机制,该机制同时也使用 HKDF 算法,在旧密钥的基础上衍生出新一轮的密钥。当加密的报文达到一定长度后,双方也需要发送 KeyUpdate 报文重新计算加密密钥。KeyUpdate 握手消息用于表示发送方正在更新其自己的发送加密密钥。任何对等方在发送 Finished 消息后都可以发送此消息。在接收 Finished 消息之前接收 KeyUpdate 消息的,实现方必须使用 "unexpected\_message" alert 消息终止连接。发送 KeyUpdate 消息后,发送方应使用新一代的密钥发送其所有流量。收到 KeyUpdate 后,接收方必须更新其接收密钥。

% 下一代 application\_traffic\_secret 计算方法如下

% \lstinputlisting[language=C,xleftmargin=2em,framexleftmargin=1.5em]{./code/key-update.txt}

% \newpage
\subsection{对称加密算法}

% 对称密码通常有两种主要形式:流密码和分组密码。流密码采用固定大小的密钥并使用它来创建任意长度的伪随机数据流,称为密钥流。 要使用流密码进行加密,可以通过将密钥流的每个位与消息的相应位进行异或来获取消息并将其与密钥流合并。当解密时候,使用加密后的消息与密钥流进行异或操作,恢复明文。 纯流密码的加密有RC4,和ChaCha20,TLS 1.3 已经删除了 RC4。分组密码与流密码不同,因为它只加密固定大小的消息。 如果要加密比块大小更短或更长的消息,则必须执行一些额外的操作,对于较短的消息,必须在消息的末尾添加一些额外的数据,作为填充。 对于较长的消息,将消息拆分为密码算法可以加密的合适大小的块,最后再组合起来。或者,可以通过使用块密码加密计数器序列并将其用作流来将块密码转换为流密码。 这称为计数器模式。 使用分组密码加密任意长度数据的一种流行方式是称为密码块链接(CBC)的模式。为了防止攻击者篡改数据,加密是不够的。 数据还需要受到完整性保护。 对于CBC模式密码,使用消息验证代码来完成消息完整性。 强的MAC具有以下特性:除非你知道密钥,否则找到与输入匹配的MAC值几乎是不可能的。 有两种方法可以组合MAC和CBC模式加密后的密文。 先加密然后计算密文的 MAC 附加到密文末尾,或者先计算明文的 MAC,然后把 MAC 附加到明文的末尾,然后加密明文和 MAC。 在TLS中,选择了后者,MAC-then-Encrypt,结果证明是错误的选择。导致 BEAST 漏洞,以及一系列填充 oracle 漏洞,例如 Lucky 13 和 Lucky Microseconds。AEAD 是 TLS 1.3 唯一保留的对称加密算法。它将完整性校验和数据加密两种功能集成在同一算法中完成。AEAD也是一种范式,不是一种具体的加密标准. AEAD是指同时包含了加密和完整性哈希的加密算法,可以由单纯的CBC和SHA1组合而成,也可以直接是GCM这种内涵了加密算法和MAC算法的范式套件直接组成。GCM的完全胜出,使得AEAD的发展方向非常明朗。GCM是一种认证加密方式,不是一种特定的加密算法。GCM 伽罗瓦/计数器模式 G就是指GMAC,C就是指CTR。 GCM 中使用 AES 等 128 位比特分组密码的 CTR 模式,并使用一个反复进行加法和乘法运算的散列函数来计算 MAC 值。CTR 模式加密与 MAC 值的计算使用的是相同密钥,所以密钥管理很方便。GCM可以提供对消息的加密性,完整性和认证性,另外,它还可以提供附加消息的完整性校验。执行对称加密和解密的算法称为对称密码。

% ChaCha20-Poly1305的主要竞争对手是基于AES-GCM的密码套件。最广泛使用的AES-GCM使用带有128位密钥的AES,但在安全性方面,AES-256与ChaCha20更具可比性。实际上,这意味着许多连接永远不会达到TLS记录的最大大小(16KB),而是使用明显更小的记录(低于1400字节)。随着连接的进行,记录大小会动态增长,扩展到大约4KB,最终扩展到16KB。大多数消息也不能精确地适合记录,并且所有大小都是可能的。

% 与AES-CTR类似,ChaCha20是一种流密码。

TLS 1.3 中使用的 AEAD 算法有 AES-256-GCM,AES-128-GCM 和 ChaCha20-Poly1305,AEAD 算法操作有4个输入:1. 要加密的明文、2. 密钥、3. 一个独特的初始化值 - IV。 在使用相同密钥调用加密操作之间必须是唯一的,否则密码的保密性将完全受到损害、4. 可选部分,一些其他非秘密的附加数据,此数据不会被加密,但会进行身份验证。在数据被加密之后,加密算法使用密钥(以及可选地加入IV)来生成辅助密钥。 辅助密钥用于生成AD的密钥散列,密文和每个密钥的各个长度。 ChaCha20-Poly1305中使用的散列函数是Poly1305,而在AES-GCM中,散列函数用的是GHASH。最后一步是获取哈希值并对其进行加密,生成最终的MAC(消息认证码)并将其附加到密文。解密操作与加密相反。 它采用相同的密钥和IV并生成密文和AD的MAC,类似于加密的方式。 然后它读取密文之后附加的MAC,并比较两者。 MAC值有任何差异都意味着密文或AD被篡改,并且它们应该被认为不安全而丢弃。如果两者匹配,则执行解密操作,恢复原始明文。AEAD将两种算法 - 加密算法和MAC算法组合成一个算法,具有可证明的安全性。当 AES\_GCM 被破解时,ChaCha20-Poly305 是一种候选算法。数字20表示它总共重复20轮加密操作。它从递增的计数器生成伪随机比特流,然后用明文对该流进行“异或”以对其进行加密(或者用密文进行“异或”以解密)。 

不需要提前知道明文来生成流,所以这种方法既可以非常高效又可以并行化。 ChaCha20是一个256位密钥加密算法,Poly1305可以与任何加密或未加密的消息一起使用,以生成密钥认证令牌。 这种令牌的目的是保证给定消息的完整性。对于 AES 块加密算法,在某些硬件上使用 AES-NI 加速指令,可以运行非常快,如果没用加速指令,单纯使用软件运行,性能会很低。而流密码 ChaCha20,则相反,软件实现性能很高,由于大部分移动设备没有 AES-NI 加速指令,运行 AES 会比较慢。ChaCha20-Poly1305 流密码算法来了,除了安全性外,它在移动设备上运行的性能较高。

% AEAD也有一定局限性:使用同一密钥加密的明文达到一定长度后,就不能再保证密文的安全性。因此,TLS 1.3中引入了密钥更新机制,一方可以(通常是服务器)向另一方发送Key Update(KU)报文,对方收到报文后对当前会话密钥再使用一次HKDF,计算出新的会话密钥,使用该密钥完成后续的通信。

CHACHA20 流加密算法。其原理和实现大致可以分成如下两个步骤:

1.基于输入的对称秘钥生成足够长度的keystream

2.将上述keystream和明文进行按位异或,得到密文

ChaCha密钥为256位(K =(k0,k1,k2,k3,k4,k5,k6,k7,以32位密钥运行。这个输出块为512位,用于密钥流(Z),以及 与明文流进行异或运算。初始状态包含16个32位值,组成 4x4 矩阵,具有128位的常量(0x61707865,0x3320646e,0x79622d32,0x6b206574)256位的密钥(k0,k1,k2,k3,k4,k5,k6,k7),32位的计数器(c)和 96位的 nonce(N0,N1,N3)。ChaCha20一共进行20轮加密操作,10轮列操作,10轮对角线操作。初始化矩阵加 20 轮操作之后的矩阵等到密钥流矩阵,和明文进行异或运算得出密文。由于计数器是32位,理论上可以生成 2 \^ 512 bit(256GB)的密钥流,所以一般长度的信息加密完全足够。解密操作和加密操作一样,接收方与发送方生成一样密钥流矩阵,与密文异或解密得到明文。

ChaCha20-Poly1305 优势\cite{7507408},\cite{7927078}
Google 推出新的加密套件并在所有移动端的 Chrome 浏览器上优先使用原因:
ChaCha20-Poly1305 避开了现有发现的所有安全漏洞和攻击;
ChaCha20-Poly1305 针对移动端设备大量使用的ARM芯片做了优化,能够充分利用 ARM 向量指令,在移动设备上加解密速度更快、更省电;Poly1305 输出只有 16字节,更加节省带宽

\newpage